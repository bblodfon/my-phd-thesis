\PassOptionsToPackage{unicode=true}{hyperref} % options for packages loaded elsewhere
\PassOptionsToPackage{hyphens}{url}
\PassOptionsToPackage{dvipsnames,svgnames*,x11names*}{xcolor}
%
\documentclass[
  12pt,
]{book}
\usepackage{lmodern}
\usepackage{setspace}
\setstretch{1.3}
\usepackage{amssymb,amsmath}
\usepackage{ifxetex,ifluatex}
\ifnum 0\ifxetex 1\fi\ifluatex 1\fi=0 % if pdftex
  \usepackage[T1]{fontenc}
  \usepackage[utf8]{inputenc}
  \usepackage{textcomp} % provides euro and other symbols
\else % if luatex or xelatex
  \usepackage{unicode-math}
  \defaultfontfeatures{Scale=MatchLowercase}
  \defaultfontfeatures[\rmfamily]{Ligatures=TeX,Scale=1}
\fi
% use upquote if available, for straight quotes in verbatim environments
\IfFileExists{upquote.sty}{\usepackage{upquote}}{}
\IfFileExists{microtype.sty}{% use microtype if available
  \usepackage[]{microtype}
  \UseMicrotypeSet[protrusion]{basicmath} % disable protrusion for tt fonts
}{}
\makeatletter
\@ifundefined{KOMAClassName}{% if non-KOMA class
  \IfFileExists{parskip.sty}{%
    \usepackage{parskip}
  }{% else
    \setlength{\parindent}{0pt}
    \setlength{\parskip}{6pt plus 2pt minus 1pt}}
}{% if KOMA class
  \KOMAoptions{parskip=half}}
\makeatother
\usepackage{xcolor}
\IfFileExists{xurl.sty}{\usepackage{xurl}}{} % add URL line breaks if available
\IfFileExists{bookmark.sty}{\usepackage{bookmark}}{\usepackage{hyperref}}
\hypersetup{
  pdftitle={The PhD Thesis you don't deserve but you need to read},
  pdfauthor={John Zobolas},
  colorlinks=true,
  linkcolor=black,
  filecolor=Maroon,
  citecolor=Blue,
  urlcolor=blue,
  breaklinks=true}
\urlstyle{same}  % don't use monospace font for urls
\usepackage[left=4cm, right=4cm, top=2.5cm, bottom=2.5cm]{geometry}
\usepackage{longtable,booktabs}
% Allow footnotes in longtable head/foot
\IfFileExists{footnotehyper.sty}{\usepackage{footnotehyper}}{\usepackage{footnote}}
\makesavenoteenv{longtable}
\usepackage{graphicx,grffile}
\makeatletter
\def\maxwidth{\ifdim\Gin@nat@width>\linewidth\linewidth\else\Gin@nat@width\fi}
\def\maxheight{\ifdim\Gin@nat@height>\textheight\textheight\else\Gin@nat@height\fi}
\makeatother
% Scale images if necessary, so that they will not overflow the page
% margins by default, and it is still possible to overwrite the defaults
% using explicit options in \includegraphics[width, height, ...]{}
\setkeys{Gin}{width=\maxwidth,height=\maxheight,keepaspectratio}
\usepackage[normalem]{ulem}
% avoid problems with \sout in headers with hyperref:
\pdfstringdefDisableCommands{\renewcommand{\sout}{}}
\setlength{\emergencystretch}{3em}  % prevent overfull lines
\providecommand{\tightlist}{%
  \setlength{\itemsep}{0pt}\setlength{\parskip}{0pt}}
\setcounter{secnumdepth}{5}
% Redefines (sub)paragraphs to behave more like sections
\ifx\paragraph\undefined\else
  \let\oldparagraph\paragraph
  \renewcommand{\paragraph}[1]{\oldparagraph{#1}\mbox{}}
\fi
\ifx\subparagraph\undefined\else
  \let\oldsubparagraph\subparagraph
  \renewcommand{\subparagraph}[1]{\oldsubparagraph{#1}\mbox{}}
\fi

% set default figure placement to htbp
\makeatletter
\def\fps@figure{htbp}
\makeatother

% for figures to fit (sometimes works well)
%\renewcommand{\textfraction}{0.05}
%\renewcommand{\topfraction}{0.8}
%\renewcommand{\bottomfraction}{0.8}
%\renewcommand{\floatpagefraction}{0.75}

% removes section headings from the top of pages
\pagestyle{plain}
% https://github.com/rstudio/rmarkdown/issues/337
\let\rmarkdownfootnote\footnote%
\def\footnote{\protect\rmarkdownfootnote}

% https://github.com/rstudio/rmarkdown/pull/252
\usepackage{titling}
\setlength{\droptitle}{-2em}

\pretitle{\vspace{\droptitle}\centering\huge}
\posttitle{\par}

\preauthor{\centering\large\emph}
\postauthor{\par}

\predate{\centering\large\emph}
\postdate{\par}

\title{The PhD Thesis you don't deserve but you need to read}
\usepackage{etoolbox}
\makeatletter
\providecommand{\subtitle}[1]{% add subtitle to \maketitle
  \apptocmd{\@title}{\par {\large #1 \par}}{}{}
}
\makeatother
\subtitle{You absolutely don't want to read this!}
\author{John Zobolas}
\date{Last updated: 03 December, 2019}

\begin{document}
\maketitle

{
\hypersetup{linkcolor=}
\setcounter{tocdepth}{1}
\tableofcontents
}
\listoftables
\listoffigures
\hypertarget{intro}{%
\chapter*{Intro}\label{intro}}
\addcontentsline{toc}{chapter}{Intro}

Chapters are currently split as:

\begin{itemize}
\tightlist
\item
  Work I have done (see \protect\hyperlink{work}{Chapter 1})
\item
  Future plans (see \protect\hyperlink{plans}{Chapter 2}). This includes the list of papers for my PhD.
\item
  For more experimental/future ideas see \protect\hyperlink{ideas}{Chapter 3}.
\item
  Various text that I have been writing here and there \protect\hyperlink{text}{Chapter 4}.
\end{itemize}

\hypertarget{introduction}{%
\subsection*{Introduction}\label{introduction}}
\addcontentsline{toc}{subsection}{Introduction}

Plan???:

Literature =\textgreater{} Curation =\textgreater{} Causal Statements =\textgreater{} Models =\textgreater{} Predicting Synergies =\textgreater{} Finding mechanisms

\hypertarget{about-the-title}{%
\subsection*{About the title}\label{about-the-title}}
\addcontentsline{toc}{subsection}{About the title}

Original title in my PhD plan was:

\textbf{Software implementations allowing new approaches toward data analysis, modeling and integration / curation of biological knowledge for Systems Medicine}

I am thinking that we may need to change it a little bit.
The reason: it started as something very general and abstract and in the end it seems I did some specific things but in various areas which still remain abstractly connected.
Alternative titles that I am thinking of currently:

\begin{itemize}
\tightlist
\item
  How to Engineer your way through a Systems Medicine PhD thesis!
\item
  Software Engineering enables optimized Systems Medicine approaches toward data analysis, modeling and curation of biological knowledge
\end{itemize}

And some for fun:

\begin{itemize}
\tightlist
\item
  Neural transformations and hydrothermal aperture in deterministic radar hydrothermal decompositions
\item
  Hydro-thermal applications and kinematic eigenvalues in exponential reliability inflationary amplitudes
\item
  Non-isothermal trellises of time-varying transmembrane and quantitative poly-and-mono-phonemes in locally capacitive hypermultiplets
\end{itemize}

\textbf{To be decided what will be the end title!}

\hypertarget{keywords}{%
\subsection*{Keywords}\label{keywords}}
\addcontentsline{toc}{subsection}{Keywords}

\emph{curation/knowledge management, VSM, causal statements, DrugLogics pipeline
(model parameterization/calibration and prediction of synergistic drug combinations, performance optimization), biomarker analysis, synergy assessment}

\begin{center}\includegraphics[width=0.5\linewidth]{img/NTNU-logo} \end{center}

\hypertarget{work}{%
\chapter{PhD work}\label{work}}

This is a summary of all the work that I have done in my PhD until now.
(mainly it's about software implementations related to the core technologies
within the group). \textbf{To include in the thesis text}.

Note though that not all of these will be part of the main thesis (maybe include the
rest in a section like `Funny PhD side-quests').

\hypertarget{dlpipe}{%
\section{Druglogics Pipeline}\label{dlpipe}}

\begin{itemize}
\tightlist
\item
  Lots of refactoring to increase the readability, maintainability and
  extendability of the source code (complete restructure of classes, addition of
  others).
  This has \textbf{RRI extensions}, because cleaning and re-structuring software code has a social aspect to it in the sense that other people can now contribute more easily, extend the code, use it (user perspective can bring changes and further improvements to software pipeline even though they may be used for research purposes) - how can you expect users to actually use a piece of code when it's not substantially documented and it's internal logics made obscure because nobody gave attention to detail and structure? How can anybody care for a (software and any) product that you have not cared enough so as to present it in an way that is acceptable, managable and proper?
\item
  Bug fixing
\item
  Enable maven packaging for easier source compilation, testing,
  installation, management and executing of the code
\item
  Added tests to modules \texttt{gitsbe} and \texttt{drabme} using JUnit5, mockito and assertJ libraries
\item
  Source code documentation + proper README files on \texttt{gitsbe}, \texttt{drabme} and \texttt{druglogics-synergy} modules
\item
  Enabling \emph{parallel simulations} in Gitsbe (performance optimization)
\item
  Added support for many features (ongoing work - see \href{https://docs.google.com/document/d/1OUupR0b-28YB9pVAww77RMecnFN6A39MYjXMjljmvG4/edit?usp=sharing}{dev\_plan\_doc})
\item
  \href{https://github.com/bblodfon/druglogics-roc-generator}{druglogics-roc-generator}:
  R shiny app to assess the performance of the Drabme results in the form of a
  ROC curve
\item
  Export support using \href{https://github.com/colomoto/bioLQM}{BioLQM}: the
  initial model + best generation models can now be exported through configuration
  options to \textbf{GINML, SBML-Qual and BoolNet} community formats
\end{itemize}

\hypertarget{vsm-dict}{%
\section{VSM}\label{vsm-dict}}

\textbf{Building VSM-dictionaries} in order to connect/translate the data from various databases and ontology providers to proper VSM-terms.
Most of this work is done in order to support Vasundra's \href{https://vtoure.github.io/causalBuilder/}{causalBuilder Tool} which is the first application of VSM after SciCura v1.

The vsm-dictionaries (code + documentation) can be found on the \href{https://github.com/vsmjs}{VSM Github page}.
They translate to VSM-terms data from BioPortal, UniProt, Ensembl, EnsemblGenomes, RNACentral, ComplexPortal and Noctua Entity Ontology.
We have also released the respective packages on \href{https://www.npmjs.com/}{npmjs}.
See for example the \href{https://www.npmjs.com/package/vsm-dictionary-bioportal}{npm package for BioPortal}.

\hypertarget{psicquic}{%
\section{PSICQUIC}\label{psicquic}}

My work at the EBI with IntAct and Noemi Del Toro to extend the PSICQUIC web service to support the
miTab 2.8 data format/standard. See the \href{http://psicquic.github.io/MITAB28Format.html}{psicquic doc} and the casualTab paper (Perfetto et al. \protect\hyperlink{ref-Perfetto2019}{2019}).

I also worked with Noemi on the update of the \href{https://github.com/MICommunity/psi-jami}{JAMI}
library to also support miTab 2.8 - this is the culmination of results from the
\href{https://2018.biohackathon-europe.org/}{BioHackathon 2018, in Paris} and the
\href{https://github.com/GREEKC/hackathon-marseille/tree/master/project_descriptions/causal_psicquic}{Marseille GREEKC hackathon event}.

\hypertarget{others}{%
\section{Others}\label{others}}

\begin{itemize}
\tightlist
\item
  Java Client for RSAT tool \href{https://github.com/bblodfon/rsat-rest-java-clients}{fetch-sequences}
\end{itemize}

\hypertarget{plans}{%
\chapter{PhD Tasks and Plans}\label{plans}}

\hypertarget{programming-tasks-todo}{%
\section{Programming Tasks TODO}\label{programming-tasks-todo}}

Tasks that I have promised that I will do to different people within the group.
These tasks enable other workflows/collaborations, etc. so they are very
important to finish before I move on to other work. You see only what's left of
those:

\begin{itemize}
\tightlist
\item
  Pipeline (see the \href{https://docs.google.com/document/d/1OUupR0b-28YB9pVAww77RMecnFN6A39MYjXMjljmvG4/edit?usp=sharing}{dev\_plan\_doc} for what is left). Most important:

  \begin{itemize}
  \tightlist
  \item
    Full BioLQM support: stable state calculation and trap spaces
  \item
    Do comparison between Aurelien's BioLQM stable state algorithm and
    BNReduction using M2 or without (Asmund already says that it BNReductions is faster
    but it's good to prove it once again)
  \end{itemize}
\item
  VSM

  \begin{itemize}
  \tightlist
  \item
    Make the \texttt{vsm-pub-dictionaries} module
  \end{itemize}
\end{itemize}

\hypertarget{papers}{%
\section{Papers}\label{papers}}

Note that the titles and the details for each paper are liable to change though the core ideas behind should not.

The papers dictate my future work for this PhD (and in that order!).

\hypertarget{paper-i-emba---an-r-package-for-ensemble-boolean-model-biomarker-analysis}{%
\subsection*{\texorpdfstring{Paper I: \texttt{emba} - an R package for ensemble boolean model biomarker analysis}{Paper I: emba - an R package for ensemble boolean model biomarker analysis}}\label{paper-i-emba---an-r-package-for-ensemble-boolean-model-biomarker-analysis}}
\addcontentsline{toc}{subsection}{Paper I: \texttt{emba} - an R package for ensemble boolean model biomarker analysis}

\hypertarget{authors}{%
\subsubsection*{Authors}\label{authors}}
\addcontentsline{toc}{subsubsection}{Authors}

John, Asmund

\hypertarget{idea}{%
\subsubsection*{Idea}\label{idea}}
\addcontentsline{toc}{subsubsection}{Idea}

This whole thing started when we questioned the predictive performance of the models generated by \texttt{Gitsbe}.
What kind of insights can we get from such a dataset by looking at each individual model's boolean equations, stable states and predictive performance?
How can we take back such knowledge and use it in order to understand more about how to generate better models in our pipeline?
How can we analyse each model's data to find nodes whose activity state or boolean model parameterization affects the manifestation of specific observed synergies?
These questions and more of the same kind lead to a large data exploration and analyses, me writing a lot of \texttt{R} code, which I ended up splitting to two packages: (J. Zobolas \protect\hyperlink{ref-R-usefun}{2019}\protect\hyperlink{ref-R-usefun}{b}) and (J. Zobolas \protect\hyperlink{ref-R-emba}{2019}\protect\hyperlink{ref-R-emba}{a}).

The idea behind the \texttt{emba} R package is to have simple functions that will help us analyse the models produced by \texttt{Gitsbe} in order to find important nodes (biomarkers) responsible for either better performance (based on a metric score like MCC) or for specific synergy(ies) prediction.

\hypertarget{what-might-come-of-this}{%
\subsubsection*{What might come of this?}\label{what-might-come-of-this}}
\addcontentsline{toc}{subsubsection}{What might come of this?}

\begin{itemize}
\item
  The R package emba (J. Zobolas \protect\hyperlink{ref-R-emba}{2019}\protect\hyperlink{ref-R-emba}{a}) is publishable by itself as an \textbf{application note paper}, but we decided with Asmund that is best to present it with an analysis on some dataset to show its use.
  For example, the package is used for analyses that will be included in Asmund's paper(s), e.g.~\emph{AGS Story: Part I} among others.
\item
  Another idea is to compare Machine Learning results with my method (on cascade/atopo results of the pipeline paper or other).
  Paper could be titled something along the lines of \textbf{``Ensemble model analysis vs Machine Learning for unraveling drug synergy mechanisms''}.
\item
  Another idea here is the results of the project \textbf{Optimize the predictive performance of the Druglogics pipeline}.
  One of the research questions here is about the \textbf{identification of optimal training data size and included nodes which are essential for good performance} (with Eirini, I am leading it).
\end{itemize}

\hypertarget{paper-ii-vsm-dictionaries-common-access-to-biological-dictionaries}{%
\subsection*{\texorpdfstring{Paper II: \emph{VSM-dictionaries}: common access to biological dictionaries}{Paper II: VSM-dictionaries: common access to biological dictionaries}}\label{paper-ii-vsm-dictionaries-common-access-to-biological-dictionaries}}
\addcontentsline{toc}{subsection}{Paper II: \emph{VSM-dictionaries}: common access to biological dictionaries}

\hypertarget{authors-1}{%
\subsubsection*{Authors}\label{authors-1}}
\addcontentsline{toc}{subsubsection}{Authors}

John, Steven, Vasundra, Martin

\hypertarget{ideaimplementation}{%
\subsubsection*{Idea/Implementation}\label{ideaimplementation}}
\addcontentsline{toc}{subsubsection}{Idea/Implementation}

A short \textbf{application note} paper for my work on \protect\hyperlink{vsm-dict}{VSM-dictionaries}.

\hypertarget{synergy-paper-ideas}{%
\section{Synergy Paper Ideas}\label{synergy-paper-ideas}}

\hypertarget{synergy-paper-i}{%
\subsection*{Synergy Paper I}\label{synergy-paper-i}}
\addcontentsline{toc}{subsection}{Synergy Paper I}

\hypertarget{title}{%
\subsubsection*{Title}\label{title}}
\addcontentsline{toc}{subsubsection}{Title}

Extending \texttt{SynergyFinder} for the use of multiple reference models for the assement of synergy in screening datasets.
A computational/mathematical paper.

\hypertarget{idea-1}{%
\subsubsection*{Idea}\label{idea-1}}
\addcontentsline{toc}{subsubsection}{Idea}

The core idea here is to extend an existing R package (Ianevski et al. \protect\hyperlink{ref-Ianevski2017}{2017}) for calculating synergy reference models in order to include Wim's generalized Bliss method and the mean synergy score by Simone Laderer (Lederer, Dijkstra, and Heskes \protect\hyperlink{ref-Lederer2018}{2018})!
Then I will test all the null reference models (Loewe, Bliss, ZIP + others) on dose-response matrix datasets (could be from Ladere's paper, from Asmund's paper, the SINTEF dataset) and see which is best at finding the synergies in each dataset.

Also, I should investigate if my own idea for a mathematical formulation of the volume-based synergy score as general method for describing 3-wise or more combinations as synergistic, could be part of this implementation.

\hypertarget{synergy-paper-ii}{%
\subsection*{Synergy Paper II}\label{synergy-paper-ii}}
\addcontentsline{toc}{subsection}{Synergy Paper II}

\hypertarget{title-1}{%
\subsubsection*{Title}\label{title-1}}
\addcontentsline{toc}{subsubsection}{Title}

On eye-balling synergy mechanisms: What is a \sout{mountain} synergy?

\hypertarget{idea-2}{%
\subsubsection*{Idea}\label{idea-2}}
\addcontentsline{toc}{subsubsection}{Idea}

I had the idea of writing a small paper that describes the \emph{eye-balling} or \emph{visual inspection} technique that is used so much in computational Biology and Medicine.
It is used pretty much in any paper I have seen but nobody has actually defined or named it.

\begin{itemize}
\tightlist
\item
  A characteristic example is \emph{eye-balling} synergies from dose-response curves, like we did in our group for the SINTEF screen data (Flobak et al. \protect\hyperlink{ref-Flobak2019}{2019}).
\item
  Another example is the thresholds that data analysts put when defining output to classifiers or the parameterization that is used and the general human intuition/engineering that is shared in all these.
\end{itemize}

As Asmund once said:

\begin{quote}
What is a mountain? What is a synergy?
\end{quote}

\begin{itemize}
\tightlist
\item
  Asmund proposed that we should contact many people to curate large drug combination screens (various datasets) and combine this with Martin's idea of \textbf{creating curation guidelines for drug combination screening}.
\end{itemize}

\hypertarget{part-miscellaneous-stuff}{%
\part*{Miscellaneous Stuff}\label{part-miscellaneous-stuff}}
\addcontentsline{toc}{part}{Miscellaneous Stuff}

\hypertarget{ideas}{%
\chapter{PhD ideas}\label{ideas}}

Several ideas that I may do or not in my PhD but I still keep here for my future
investigations!

\hypertarget{quantum}{%
\section{Quantum logic formalism}\label{quantum}}

My favourite! Investigate if instead of a logical modeling formalism, the idea
of (quantum) logical gates can be used to represent and analyse protein interaction
networks.
The \textbf{core idea} makes sense: you don't know the state of a protein, but when
you measure it, only then you really know what it is.

May also be worth to look at a \href{https://doi.org/10.1007/11885191_18}{game-theoritic approach}
to find attractors and such.

\hypertarget{comp}{%
\section{Compare fixpoint tools}\label{comp}}

Idea: Compare different tools that calculate fixpoints for logical modeling.
Faster wins of course :)

Models used for testing could be of different types:

\begin{itemize}
\tightlist
\item
  self-contained
\item
  varying the number of input nodes (1-n)
\item
  small to large number of nodes
\item
  small to large number of edges
\item
  scale-free (boolnet generated) vs random (varying K connectivity)
\item
  play with form of the boolean equations
\item
  others ???
\end{itemize}

Workflow for this includes:

\begin{itemize}
\tightlist
\item
  support BNReduction data format by \href{https://doi.org/10.1186/1471-2105-15-221}{Veliz-Cuba}
  in BioLQM
\item
  add support for calculating the fixpoints using the Colomoto docker (python
  interface) + BNReduction
\item
  then comparison between \textbf{BioLQM, Pint, MABOSS and BNReduction} could be done then
  in a Jupiter colomoto-enabled notebook!
\end{itemize}

Further extension/comparisons could be:

\begin{itemize}
\tightlist
\item
  (Akutsu, Hayashida, and Tamura \protect\hyperlink{ref-Akutsu2009}{2009}) - Integer programming method
\item
  (Dubrova and Teslenko \protect\hyperlink{ref-Dubrova2009}{2009}) - SAT-based
\end{itemize}

\hypertarget{reasoning-with-vsm}{%
\section{Reasoning with VSM}\label{reasoning-with-vsm}}

The idea here is to use VSM to annotate sentences about some knowledge area,
store this information in a format like RDF or something else (graph database?)
and then ask questions that will enable you to learn stuff that you didn't know
before.

One goal would be to show the superiority of \emph{connection-based reasoning}
(humans and what VSM encapsulates) vs \emph{logic reasoning} (OWL).

Another thing I thought was to just translate the VSM-data to PROLOG
and then ask questions using that logical language framework. It is a way to show
that you \emph{learned} something using the VSM-supported curation but I don't know
where to go from there\ldots{} this whole knowledge semantics and reasoning stuff
seem to be a PhD on its own :) There are a lot of things that should
be investigated for this idea to materialize properly (lots of reading).

An idea by Steven:

Mapping / inferring different forms of representing interactions (molecular
and/vs.~causal) using a VSM sentence presentation as a graph diagram. A first
step to reasoning would be to make some rules like, if you have VSM-sentences
A and B, then you can infer C from that.

\hypertarget{use-logical-modeling-to-predict-single-drug-data}{%
\section{Use Logical modeling to predict single-drug data}\label{use-logical-modeling-to-predict-single-drug-data}}

Asmund's proposal idea that he sent to my email once. Has to do about \emph{mechanistic drug response prediction analysis}:

\begin{itemize}
\tightlist
\item
  Automate drug target profile annotation from:

  \begin{itemize}
  \tightlist
  \item
    (Klaeger et al. \protect\hyperlink{ref-Klaeger2017}{2017})
  \item
    \href{http://www.kinase-screen.mrc.ac.uk/}{mrc ppu}
  \item
    (Davis et al. \protect\hyperlink{ref-Davis2011}{2011})
  \end{itemize}
\item
  Omics data (rna, cnv etc)

  \begin{itemize}
  \tightlist
  \item
    COSMIC
  \item
    CCLE
  \end{itemize}
\item
  Drug screen data

  \begin{itemize}
  \tightlist
  \item
    Single drug

    \begin{itemize}
    \tightlist
    \item
      COSMIC/GDSC
    \item
      CCLE
    \end{itemize}
  \item
    Combo

    \begin{itemize}
    \tightlist
    \item
      (O'Neil et al. \protect\hyperlink{ref-ONeil2016}{2016})
    \item
      (Holbeck et al. \protect\hyperlink{ref-Holbeck2017}{2017})
    \end{itemize}
  \end{itemize}
\end{itemize}

\textbf{My idea} is more like this:\\
Predict drug-response curves from drug combination datasets (GDSC, CCLE),
using logical modeling for singaling network analysis or translation from
logical to ODE modeling. Also try to predict drug combinations datasets
(dose-response matrices?). Pretty much what is done in this paper (Fröhlich et al. \protect\hyperlink{ref-Frohlich2018}{2018})
with help from (Wittmann et al. \protect\hyperlink{ref-Wittmann2009}{2009}) for converting boolean models to continuous.

\hypertarget{druglogics-pipeline-related}{%
\section{Druglogics-Pipeline related}\label{druglogics-pipeline-related}}

\hypertarget{harmony-search}{%
\subsection{Harmony Search}\label{harmony-search}}

Nice idea because it's related to music!
Investigate if \href{https://doi.org/10.1016/j.proeng.2016.07.510}{this algorithm}
could be used for optimizing the boolean equations for \texttt{gitsbe} - thus opening the
stage for \texttt{JazzLogics}.

\hypertarget{train-models-to-cell-specific-proliferation}{%
\subsection{Train models to cell-specific proliferation}\label{train-models-to-cell-specific-proliferation}}

Concept is that training models to proliferate provides a wider variance of models than the
cell-specific trained ones in \texttt{gitsbe}: main directive is \textbf{proliferation},
not just fitting to a steady state pattern. So a hybrid training approach should
be way more advantageous.

\hypertarget{a-bottom-up-model-building-for-drug-prediction}{%
\subsection{A bottom-up model building for drug prediction}\label{a-bottom-up-model-building-for-drug-prediction}}

Start with a model and some observed synergies. Build/train/produce models that
predict the first observed synergy (using Harmony Search?), from them the next
one, etc. You end up with many models that can predict all the observed
synergies or you try to find out why that cannot happen for example (e.g.~
contrasting synergies?). Do the latest models' stable states or attractors
correspond to activity of proteins from literature?

\hypertarget{simulate-cancer-resistance}{%
\subsection{Simulate cancer resistance}\label{simulate-cancer-resistance}}

For example, you have some models that predict some (observed) synergies or you
just find some synergistic drug comibnations for these models or per model.
Then, you modify these models in order to be resistant to these drugs, simulating
thus the cancer rewiring process! Then, you apply (n+1) drug combinations to
win over the resistance (and you do this procedure at more levels to suggest
3-way, 4-way drug combos and why there might be cancer models that can `win'
over these models and continue the proliferation). You end up with super cancer
resistant models and methods to achieve them or reasons why this cannot happen
at all.

\hypertarget{causal-json-or-mi-json-to-boolean-model-converter}{%
\subsection*{Causal-JSON or MI-JSON to boolean model converter}\label{causal-json-or-mi-json-to-boolean-model-converter}}
\addcontentsline{toc}{subsection}{Causal-JSON or MI-JSON to boolean model converter}

\hypertarget{idea-3}{%
\subsubsection*{Idea}\label{idea-3}}
\addcontentsline{toc}{subsubsection}{Idea}

This idea is like a continuation of the \texttt{causalBuilder} tool by Vasundra coupled with the need to have a more proper representation of complexes (and families) in our logical models (better models, better predictions).
Asmund had \emph{manually} changed some logical equations in his paper (Flobak et al. \protect\hyperlink{ref-Flobak2015}{2015}), in order to make the model more compliant with biology knowledge and literature findings.
One of them was about the beta-catenin complex and its constituents (connected with \emph{AND's} instead of \emph{OR's}) and the rest were about changing the link operators of the logical equations (from \emph{AND NOT} to \emph{OR NOT}).
The latter is something that is enabled through the mutations introduced by the genetic algorithm of \texttt{Gitsbe}.
The former depends on the dataset and the representation of complexes.\footnote{There is actually a mutation that can change this but not in the way that we want - i.e.~all components of a complex should be connected with an \emph{AND}}

Only Signor (Licata et al. \protect\hyperlink{ref-Licata2019}{2019}) has some complexes + interaction data but they are seperate files, making it thus difficult (and non-elegant computationally-wise) to integrate such knowledge/data to boolean models.
Also Vasundra's experience with Reactome data in miTab2.8 showed us the difficulty to match binary interactions to a data model flexible enough to represent complexes and their internal components. Causal-JSON and the recursive schema that we thought allows the curator to put both the complex ID and it's constituents in the same data structure.

\hypertarget{proposed-workflow}{%
\subsubsection*{Proposed Workflow}\label{proposed-workflow}}
\addcontentsline{toc}{subsubsection}{Proposed Workflow}

\begin{enumerate}
\def\labelenumi{\arabic{enumi}.}
\item
  Get interaction + complexes/families data (Signor most probably or a form of CASCADE + complexes).
  Note that for the reason I explained above miTab 2.8 is out of the question, so the Signor data I am refering to is the .tsv files they offer (interaction data, complexes, families).
  And most probably I am referring to a \textbf{pathway interaction dataset} not the whole Signor data.
  For example, the \href{https://signor.uniroma2.it/pathway_browser.php?organism=human\&pathway_list=SIGNOR-WNT\&level=1}{Wnt Signaling pathway}.
\item
  Build a small module that translates the (Signor) data to Causal-JSON.
\item
  \hypertarget{causalJSONPoint3}{}

  \textbf{Main:} Build a package that translates the causal-JSON data to a logical model with some filtering and parameterization included (e.g.~filter based on cell line (so \emph{cell-line} specific topologies), conditions on the biological state: `by phoshorylation', exp. evidence, assertion/confidence score, species, compartment). So, \textbf{causal-JSON to .bnet files (logical equations)}, while substituting/extending nicely the complexes and families.
\item
  Showcase some small application of this logical model end-product:

  \begin{itemize}
  \tightlist
  \item
    use for example the \href{https://github.com/colomoto/colomoto-docker}{colomoto notebook}, do some small trapspace analysis and show that some results from literature or from previous logical or other models can now be reproduced with a better biological representation in the model itself, AUTOMATICALLY!
  \item
    make many logical models of the pathways in Signor with simple attractor analysis and put them into the GinSim model repository for reference for the logical community.
  \item
    extend \texttt{atopo} module to use the main package (see \protect\hyperlink{causalJSONPoint3}{point No.~3}) and use it for finding drug combinations (comparing attractors or prediction results of automated topology building without complexes vs automated topology built from causal-JSON with complexes and families in each case).
    The main thing here would be of course better prediction performance results based on a better logical model representation.
    I could tweak \texttt{atopo} to choose actually not all of Signor's data but specific pathways to include in the analysis and this will help I believe to build smaller topologies for specific drug combinations that we want to test.
  \end{itemize}
\end{enumerate}

\hypertarget{text}{%
\chapter{Text}\label{text}}

Here I have various text the I write at times and I will include most probably in my completed thesis report:

\hypertarget{why-automated-topology}{%
\section{Why automated topology?}\label{why-automated-topology}}

My take on this:

\begin{enumerate}
\def\labelenumi{\arabic{enumi}.}
\tightlist
\item
  Yes, one of the reasons is advantage in the simulations in the sense that when you have logical models that have no inputs you are statistically more able to have models with a few/less (even better: 1) stable state(s).
  See it like this: if a logical model has one input (let's say node X) then it's sure that this logical model will have one attractor with X:0 and one with X:1, be it a stable state or a more complex attractor.
  Two inputs, 4 attractors, etc. (Denis helped me realise this actually on a talk in Athen's ECCB, great times)
\item
  The second reason that Asmund told me about when I asked him the same thing, was one of the most basic hypothesis behind his modeling - which sums up to \emph{where cancer comes from}.
  By using a no-inputs topology we adhere to the principle that cancer is something that relates to the system itself and not to the external interactions of the system.
  It comes from dysregulations within related to ``broken'' circuits, etc.
  It is in contrast with the traditional view on the same thing that experimentalists used to understand cancer: I perturb the system (cell) by inhibiting a specific hormone/receptor (input) and see how it reacts (and where most modeling approaches are based on).
\end{enumerate}

Asmund's take on this:
1. few stable states
2. cancer is a system disease in itself (not related to e.g.~external hormones, they are present also for healthy cells, it is something in the cancer cell that allows it to sense the external hormone differently than healthy cells).

In addition (and maybe related to point 1):
3. In `traditional' modeling a system is defined to respond in a certain way to a set of specified `inputs' (I mean not here model input nodes but rather a configuration of the model, e.g.~ERK is active).
A self-contained topology merges the input condition and the output response in a single observable entity: When the system is initialized in its stable state it will remain there.
Therefore observation of baseline signaling is both the input and the output, reduces need for perturbations.

\begin{enumerate}
\def\labelenumi{\arabic{enumi}.}
\setcounter{enumi}{3}
\tightlist
\item
  In addition to few stable states (point 1) I believe a self contained topology also means few possible parmeterizations.
  I don't have any mathematical proof of this but it seems reasonable to me.
\end{enumerate}

\hypertarget{appendix-appendix}{%
\appendix}


\hypertarget{papers-1}{%
\chapter*{Papers}\label{papers-1}}
\addcontentsline{toc}{chapter}{Papers}

Papers that are already published and I am in the list of authors:

\begin{itemize}
\tightlist
\item
  (Perfetto et al. \protect\hyperlink{ref-Perfetto2019}{2019})
\end{itemize}

Papers that will probably be published and I will probably be in the list of authors:

\begin{itemize}
\tightlist
\item
  \href{https://www.tinyurl.com/bh2018write}{The Biohackathon 2018 paper}
\item
  Asmund's papers (2)
\item
  Vasundra's causalBuilder tool paper
\end{itemize}

\hypertarget{bookdown-useful-links}{%
\chapter*{Bookdown useful links}\label{bookdown-useful-links}}
\addcontentsline{toc}{chapter}{Bookdown useful links}

\begin{itemize}
\tightlist
\item
  \href{https://github.com/rstudio/bookdown/}{Bookdown github repo}
\item
  \href{https://bookdown.org/yihui/bookdown/}{Bookdown package reference}
\item
  \href{https://eddjberry.netlify.com/post/writing-your-thesis-with-bookdown/}{Writing Thesis with Bookdown}
\item
  \href{https://arm.rbind.io/slides/bookdown.html}{Bookdown workshop slides}
\end{itemize}

\hypertarget{references}{%
\chapter*{References}\label{references}}
\addcontentsline{toc}{chapter}{References}

\hypertarget{refs}{}
\leavevmode\hypertarget{ref-Akutsu2009}{}%
Akutsu, Tatsuya, Morihiro Hayashida, and Takeyuki Tamura. 2009. ``Integer programming-based methods for attractor detection and control of boolean networks.'' In \emph{Proceedings of the 48h Ieee Conference on Decision and Control (Cdc) Held Jointly with 2009 28th Chinese Control Conference}, 5610--7. IEEE. \url{https://doi.org/10.1109/CDC.2009.5400017}.

\leavevmode\hypertarget{ref-Davis2011}{}%
Davis, Mindy I, Jeremy P Hunt, Sanna Herrgard, Pietro Ciceri, Lisa M Wodicka, Gabriel Pallares, Michael Hocker, Daniel K Treiber, and Patrick P Zarrinkar. 2011. ``Comprehensive analysis of kinase inhibitor selectivity.'' \emph{Nature Biotechnology} 29 (11): 1046--51. \url{https://doi.org/10.1038/nbt.1990}.

\leavevmode\hypertarget{ref-Dubrova2009}{}%
Dubrova, Elena, and Maxim Teslenko. 2009. ``A SAT-Based Algorithm for Computing Attractors in Synchronous Boolean Networks.'' \href{https://arxiv.org/pdf/0901.4448.pdf\%20https://ieeexplore.ieee.org/document/5958722/}{https://arxiv.org/pdf/0901.4448.pdf https://ieeexplore.ieee.org/document/5958722/}.

\leavevmode\hypertarget{ref-Flobak2015}{}%
Flobak, Åsmund, Anaïs Baudot, Elisabeth Remy, Liv Thommesen, Denis Thieffry, Martin Kuiper, and Astrid Lægreid. 2015. ``Discovery of Drug Synergies in Gastric Cancer Cells Predicted by Logical Modeling.'' Edited by Ioannis Xenarios. \emph{PLOS Computational Biology} 11 (8): e1004426. \url{https://doi.org/10.1371/journal.pcbi.1004426}.

\leavevmode\hypertarget{ref-Flobak2019}{}%
Flobak, Åsmund, Barbara Niederdorfer, Vu To Nakstad, Liv Thommesen, Geir Klinkenberg, and Astrid Lægreid. 2019. ``A high-throughput drug combination screen of targeted small molecule inhibitors in cancer cell lines.'' \emph{Scientific Data} 6 (1): 237. \url{https://doi.org/10.1038/s41597-019-0255-7}.

\leavevmode\hypertarget{ref-Frohlich2018}{}%
Fröhlich, Fabian, Thomas Kessler, Daniel Weindl, Alexey Shadrin, Leonard Schmiester, Hendrik Hache, Artur Muradyan, et al. 2018. ``Efficient Parameter Estimation Enables the Prediction of Drug Response Using a Mechanistic Pan-Cancer Pathway Model.'' \emph{Cell Systems} 7 (6): 567--579.e6. \url{https://doi.org/10.1016/J.CELS.2018.10.013}.

\leavevmode\hypertarget{ref-Holbeck2017}{}%
Holbeck, Susan L, Richard Camalier, James A Crowell, Jeevan Prasaad Govindharajulu, Melinda Hollingshead, Lawrence W Anderson, Eric Polley, et al. 2017. ``The National Cancer Institute ALMANAC: A Comprehensive Screening Resource for the Detection of Anticancer Drug Pairs with Enhanced Therapeutic Activity.'' \emph{Cancer Research}. \url{https://doi.org/10.1158/0008-5472.CAN-17-0489}.

\leavevmode\hypertarget{ref-Ianevski2017}{}%
Ianevski, Aleksandr, Liye He, Tero Aittokallio, and Jing Tang. 2017. ``SynergyFinder: a web application for analyzing drug combination dose--response matrix data.'' \emph{Bioinformatics} 33 (15): 2413--5. \url{https://doi.org/10.1093/bioinformatics/btx162}.

\leavevmode\hypertarget{ref-Klaeger2017}{}%
Klaeger, Susan, Stephanie Heinzlmeir, Mathias Wilhelm, Harald Polzer, Binje Vick, Paul-Albert Koenig, Maria Reinecke, et al. 2017. ``The target landscape of clinical kinase drugs.'' \emph{Science (New York, N.Y.)} 358 (6367): eaan4368. \url{https://doi.org/10.1126/science.aan4368}.

\leavevmode\hypertarget{ref-Lederer2018}{}%
Lederer, Simone, Tjeerd M H Dijkstra, and Tom Heskes. 2018. ``Additive Dose Response Models: Explicit Formulation and the Loewe Additivity Consistency Condition.'' \emph{Frontiers in Pharmacology} 9: 31. \url{https://doi.org/10.3389/fphar.2018.00031}.

\leavevmode\hypertarget{ref-Licata2019}{}%
Licata, Luana, Prisca Lo~Surdo, Marta Iannuccelli, Alessandro Palma, Elisa Micarelli, Livia Perfetto, Daniele Peluso, Alberto Calderone, Luisa Castagnoli, and Gianni Cesareni. 2019. ``SIGNOR 2.0, the SIGnaling Network Open Resource 2.0: 2019 update.'' \emph{Nucleic Acids Research}, October. \url{https://doi.org/10.1093/nar/gkz949}.

\leavevmode\hypertarget{ref-ONeil2016}{}%
O'Neil, Jennifer, Yair Benita, Igor Feldman, Melissa Chenard, Brian Roberts, Yaping Liu, Jing Li, et al. 2016. ``An Unbiased Oncology Compound Screen to Identify Novel Combination Strategies.'' \emph{Molecular Cancer Therapeutics} 15 (6): 1155--62. \url{https://doi.org/10.1158/1535-7163.MCT-15-0843}.

\leavevmode\hypertarget{ref-Perfetto2019}{}%
Perfetto, L, M L Acencio, G Bradley, G Cesareni, N Del Toro, D Fazekas, H Hermjakob, et al. 2019. ``CausalTAB: the PSI-MITAB 2.8 updated format for signalling data representation and dissemination.'' Edited by Jonathan Wren. \emph{Bioinformatics}, February. \url{https://doi.org/10.1093/bioinformatics/btz132}.

\leavevmode\hypertarget{ref-Wittmann2009}{}%
Wittmann, Dominik M, Jan Krumsiek, Julio Saez-Rodriguez, Douglas A Lauffenburger, Steffen Klamt, and Fabian J Theis. 2009. ``Transforming Boolean models to continuous models: methodology and application to T-cell receptor signaling.'' \emph{BMC Systems Biology} 3 (1): 98. \url{https://doi.org/10.1186/1752-0509-3-98}.

\leavevmode\hypertarget{ref-R-emba}{}%
Zobolas, John. 2019a. \emph{Emba: Ensemble Boolean Model Biomarker Analysis}. \url{https://github.com/bblodfon/emba}.

\leavevmode\hypertarget{ref-R-usefun}{}%
---------. 2019b. \emph{Usefun: A Collection of Useful Functions by John}. \url{https://github.com/bblodfon/usefun}.

\end{document}
